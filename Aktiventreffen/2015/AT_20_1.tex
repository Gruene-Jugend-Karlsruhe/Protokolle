\documentclass[10pt,a4paper]{article}
\usepackage[utf8]{inputenc}
\usepackage[T1]{fontenc}
\usepackage[ngerman]{babel}

% Art der Sitzung
\newcommand{\Sitzung}{Mitgliederversammlung u. Aktiventreffen}
\newcommand{\Autor}{Michel von Czettritz und Neuhaus}
\newcommand{\Datum}{24.6.14}
\newcommand{\Verspaetung}{10 min}
\newcommand{\Dauer}{~min}
\newcommand{\Ort}{\begin{tabular}{l}Grünes Büro,\\ Sophienstr. 58,\\ 76133 Karlsruhe\end{tabular}}
\newcommand{\Vorstand}{\begin{tabular}{c}Vorstand:\\ Annika Rudolph,\\ Jonas Kittel,\\ Leonie Wolf,\\ Lisa Merkens,\\ Michel von Czettritz und Neuhaus\end{tabular}}

\usepackage{hyperref}
\usepackage{tocloft}
\usepackage{tikz}
\usepackage{eso-pic}
\usepackage{ifthen}

\ifthenelse{\equal{\Sitzung}{Aktiventreffen}}{
	\newcommand{\Artikel}{des}
	\newcommand{\TOPnull}{Vorstellungsrunde}
	\newcommand{\Zeit}{19:30 Uhr}
}{
	\newcommand{\Artikel}{der}
	\ifthenelse{\equal{\Sitzung}{Vorstandssitzung}}{
		\newcommand{\TOPnull}{Reflexion des letzten Treffens}
		\newcommand{\Zeit}{20:00 Uhr}
	}{
		\newcommand{\TOPnull}{Formalia}
		\newcommand{\Zeit}{19:30 Uhr}
	}
}
\newcommand{\Titel}{Protokoll \Artikel~\Sitzung~der GJ Karlsruhe}

% "TOP"s
\renewcommand{\thesection}{TOP\arabic{section}:}
\renewcommand{\thesubsection}{\arabic{subsection}}
\setlength{\cftsecnumwidth}{1,4cm}
\setcounter{section}{-2}

% Hintergrundbild
\newcommand\BackgroundPic{
\put(0,0){
\parbox[b][1.02\paperheight]{\paperwidth}{
\begin{flushright}
\includegraphics[width=\textwidth,
keepaspectratio]{GJ_doc.jpg}
\vfill
\includegraphics[width=\textwidth,
keepaspectratio]{GJ_doc2.jpg}
\end{flushright}
}}}

% Titelseite
\newcommand{\Titelseite}{
\AddToShipoutPicture*{\BackgroundPic}
\begin{titlepage}
	\vspace*{\fill}
	\begin{flushright}
		Grüne Jugend Karlsruhe\\
		Sophienstraße 58\\
		76133 Karlsruhe\\
		kontakt@gruene-jugend-karlsruhe.de\\
		www.gruene-jugend-karlsruhe.de\\
	\end{flushright}
	\vspace*{2cm}
	\begin{center}
		{\fontsize{0.5cm}{0.5cm}\selectfont \textbf{\Titel}\\}
		\vspace*{1cm}
		\begin{tabular}{lcl}
			Verfasser &:& \Autor \\[0.2cm]
			Datum &:& \Datum \\
			Uhrzeit &:& \Zeit + \Verspaetung \\
			Dauer &:& \Dauer \\[0.3cm]
			Ort &:& \Ort
		\end{tabular}
		\vspace*{4cm}
		\begin{center}
			\Vorstand
		\end{center}
	\end{center}
	% Legende
	\begin{table}[b]
		\centering
		\begin{tabular}{cl}
			$(\,\cdot\,)$ & Kommentar des Autors \\
			$[\,\cdot\,]$ & Verantwortliche Personen \\ 
		\end{tabular}
	\end{table}
	\vspace*{\fill}
\end{titlepage}
}


\begin{document}

\Titelseite

\tableofcontents
\newpage


\section{Vorstellungsrunde}

\section*{Vorbereitung MV}
Das Präsidium stellt die Beschlussfähigkeit fest. Das Präsidium hat Aljosha inne.

\section{Nachwahl Vorstand}
Es lassen sich Miriam und Caro als Kandidatinnen für den Frei werdenden Vorstandsposten aufstellen. Die beiden Stellen sich selber in ein paar Sätzen vor.\\
Ali stellt das Wahlverfahren vor. \\
\subsection*{Anwesende Personen}
12 Stimmberechtigte.\\
Quorum: 7 stimmen im erste Wahlgang. 

\subsection*{Erster Wahlgang}
12 abgegeben alle gültig

\begin{description}
\item[4] Miriam 
\item[5] Caro
\item[2] Enthaltung
\item[1] Nein
\end{description}

\subsection*{Zweiter Wahlgang}
12 abgegeben alle gültig

\begin{description}
\item[4] Miriam
\item[7] Caro
\item[1] Enthaltungen
\item[0] Nein
\end{description}

Daraus folgt Caro ist gewählt. Sie nimmt die Wahl an.

\section{GO-Anträge}
Jonas stellt einen GO Antrag. für den Antrag siehe Anhang. Es wird geduldet gegen ermöglicht ersetzt. 

\begin{description}
\item[9] Ja Stimmen
\item[1] Nein Stimme
\item[1] Enthaltung
\end{description}

\footnote{Michel Updated die GO}

\section{Bericht Aktionsbündnis WLAN}
Es gibt die Idee Freifunk für Flüchtlinge in die LEA zu bringen. Die Gruppe Trifft sich morgen ab 19 Uhr im Entropia.

\section{Neujahrsempfang}
Der Neujahrsempfang ist dieses Wochenende. Man sollte sich über die GJBW anmelden. Ali Stellt das Programm kurz vor. Es melden sich 8 - 10 menschen die kommen wollen.

\section{Termine}
\begin{description}
\item [Morgen ab 17.30] is wieder Netzwerk gegen Rechtstreffen
\item [Morgen ab 19:00] ist FFfFKA Treffen im Entropia
\item [Dieses Wochenende] ist Neujahrsempfang
\item [Nächsten Montag] Anti Kagida Demo 17 Uhr Stefansplatz
\end{description}


\section{Bericht Gemeinderat}
Zoe berichtet aus dem Gemeinderat. Es geht mehrheitlich um den Doppelhaushalt. 

\section{Aktions Bündnis gegen Kagida}
Die GJKA ist dafür dieses Aktionsbündnis zu unterstützen.

\section{Primark}
Clemens möchte eine Aktion gegen Primark starten. Es Bricht eine Debatte aus warum genau Primark schlimmer ist als andre Discounter.

\end{document}
